\documentclass[conference]{IEEEtran}
\IEEEoverridecommandlockouts

\usepackage{cite}
\usepackage{amsmath,amssymb,amsfonts}
\usepackage{algorithmic}
\usepackage{graphicx}
\usepackage{textcomp}
\usepackage{xcolor}
\def\BibTeX{{\rm B\kern-.05em{\sc i\kern-.025em b}\kern-.08em
    T\kern-.1667em\lower.7ex\hbox{E}\kern-.125emX}}
\begin{document}

\title{Plane Chat: Technical Report\\
{\footnotesize Project for the course \textit{Wireless Networks For Mobile Applications}}
}

\author{\IEEEauthorblockN{1\textsuperscript{st} Gianluca Bresolin}
\IEEEauthorblockA{\textit{University of Padua: Computer Science} \\
\textit{University of Padua}\\
Padua, Italy \\
gianbreso02@gmail.com}
}

\maketitle

\begin{abstract}
Plane Chat abstract.
\end{abstract}

\begin{IEEEkeywords}
Wireless Ad-Hoc Networks, Multipeer Connectivity, DSDV Protocol, iOS
\end{IEEEkeywords}

\section{Introduction}
Wireless networks have become an integral part of our daily lives, enabling
communication and connectivity in various environments. Those networks have
gain a significant importance over the years, especially due to the widespread
adoption of mobile devices, with applications that rely on wireless
communication to provide services and functionality. \\

In relation to wired networks, wireless networks present different and unique
challenges and characteristics that mainly depend on the scenario in which they
are deployed. Those challenges require the design and implementation of
specialized protocols and technologies to ensure network operation and
performance. \\
Specifically, \textit{ad-hoc} wireless networks do not require any
pre-existing infrastructure and allow the construction of termporary networks
with no wires and no administrative intervention required. Ad-hoc networks
differ significantly from existing wired networks: the lack of physical
connections between devices indeed allows for greater mobility and flexibility,
which makes the topology of interconnections dynamic and constantly changing,
with nodes that have to discover where others are. 
Moreover, since users will not wish to perform any administrative actions to
set up such networks, we do not assume that every device is within communication
range of every other device. Those characteristics make the design of protocols
for ad-hoc networks particularly challenging, as they must be able to adapt to
the dynamic nature of the network and to the possible lack of complete
connectivity. \\

In this report, to investigate the design and implementation challenges of ad-hoc
wireless networks, we present \textit{Plane Chat}, a mobile \textit{iOS}
application enabling users to create groups and chat with each other during
flights. Given the inherent constraints of such environments, the app is
designed to work in a situation where devices are in close proximity without
relying on pre-existing infrastructure, such as Wi-Fi access points or cellular 
networks. Therefore, the app facilitate group creation and message exchange by
instantiating peer-to-peer connections via the \textit{Multipeer Connectivity}
framework provided by \textit{Apple}, thereby creating an ad-hoc network. \\

To overcome the \textit{Multipeer Connectivity} limitation of having a maximum of
eight devices per session and enhance network scalability in order to allow more
users to join the same group and chat with each other, the application leverages
the framework as a link-layer substrate. Consequently, we implement the
\textit{Destination-Sequenced Distance-Vector} (DSDV) routing protocol at the
network layer to facilitate multi-hop communication between devices. \\
At the application layer, the message dissemination follows a "best-effort"
delivery model. To maintain low overhead and simplicity, the protocol does not
implement any acknowledgement (ACK) mechanism or retransmission logic.
Consequently, the application operates under an unreliable communication
paradigm, where the delivery of packets to all nodes is not stricly guaranteed.
\\

The remainder of this report is organized as follows: Section
\ref{sec:PlaneChat} provides an overview of the \textit{Plane Chat} application,
including its features and functionalities. Section
\ref{sec:DesignAndImplementation} describes the design and implementation of the
system. Section \ref{sec:ExperimentalSetupAndResults} presents the experimental
setup and results obtained from testing the application. Finally, Section
\ref{sec:Conclusions} concludes the report and discusses potential future work
and improvements for the \textit{Plane Chat} application. 

\section{PlaneChat}
\label{sec:PlaneChat}

\section{System Design and Implementation}
\label{sec:DesignAndImplementation}

\subsection{Multipeer Connectivity Framework}
\subsection{DSDV Protocol}

\section{Experimental Setup and Results}
\label{sec:ExperimentalSetupAndResults}

\section{Conclusions}
\label{sec:Conclusions}

% \begin{table}[htbp]
% \caption{Table Type Styles}
% \begin{center}
% \begin{tabular}{|c|c|c|c|}
% \hline
% \textbf{Table}&\multicolumn{3}{|c|}{\textbf{Table Column Head}} \\
% \cline{2-4} 
% \textbf{Head} & \textbf{\textit{Table column subhead}}& \textbf{\textit{Subhead}}& \textbf{\textit{Subhead}} \\
% \hline
% copy& More table copy$^{\mathrm{a}}$& &  \\
% \hline
% \multicolumn{4}{l}{$^{\mathrm{a}}$Sample of a Table footnote.}
% \end{tabular}
% \label{tab1}
% \end{center}
% \end{table}

% \begin{figure}[htbp]
% \centerline{\includegraphics{fig1.png}}
% \caption{Example of a figure caption.}
% \label{fig}
% \end{figure}

\section*{References}
% Please number citations consecutively within brackets \cite{b1}. 

% \begin{thebibliography}{00}
% \bibitem{b1} G. Eason, B. Noble, and I. N. Sneddon, ``On certain integrals of Lipschitz-Hankel type involving products of Bessel functions,'' Phil. Trans. Roy. Soc. London, vol. A247, pp. 529--551, April 1955.
% \bibitem{b8} D. P. Kingma and M. Welling, ``Auto-encoding variational Bayes,'' 2013, arXiv:1312.6114. [Online]. Available: https://arxiv.org/abs/1312.6114
% \bibitem{b9} S. Liu, ``Wi-Fi Energy Detection Testbed (12MTC),'' 2023, gitHub repository. [Online]. Available: https://github.com/liustone99/Wi-Fi-Energy-Detection-Testbed-12MTC
% \bibitem{b10} ``Treatment episode data set: discharges (TEDS-D): concatenated, 2006 to 2009.'' U.S. Department of Health and Human Services, Substance Abuse and Mental Health Services Administration, Office of Applied Studies, August, 2013, DOI:10.3886/ICPSR30122.v2
% \end{thebibliography}

\end{document}
